\documentclass[lang=cn,11pt,chinese,base=hide]{elegantbook}

\title{洛阳理工学院校园网使用指南}
\subtitle{LuoYang Institute Of Science And Technology The CN Description}

\institute{Ajia Network Technology Studio}
\date{April 12, 2021}
\author{AjiaErin}
\version{1.0.0}

\extrainfo{Your college can be wonderful. --- AjiaErin}

\logo{logo.png}
\cover{bg1.jpg}

% 本文档命令
\usepackage{array}
\newcommand{\ccr}[1]{\makecell{{\color{#1}\rule{1cm}{1cm}}}}
% 修改目录深度
\setcounter{tocdepth}{2}

\begin{document}

\maketitle
\frontmatter

\chapter*{特别声明}
\markboth{Introduction}{前言}

\begin{center}
\underline{写这个特别声明的主要目的是:为了让你能将这篇文档看完!!!}
\end{center}

写这篇文档,我没有带着丝毫“看轻”或者“鄙视”提问者的心态(或许言语极为偏激,文风如此,
接受不了现在可以关了文档)。首先我,以及很多潜在的乐意帮助你解决关于校园网问题的同学们,
都能够理解你因为一段断网or没网纠结住的急切心情,因为我们都是从新手一路走来,也遇到过和你一样的情况。

本文档面向小白,(如有错误及优化建议,还请巨佬不吝斧正!)写这篇文档的目的,是为了帮助使用校园网的人,
能够更加高效地自行解决问题,或提出自己的问题并获得最大可能的解答。这有两个好处,
首先自然是解决了提问者的问题,其次是一个“好问题”能够帮助遇到同样问题的后来人更简便地明白问题的原委并解决问题。

所以,我希望阅读到此的人,能够以友好的态度继续阅读下去。

本文档只以PDF呈现,下载到本地后断网也可以查看,文档我会开源在GitHub,并会提供多个下载地址!


\vskip 1.5cm

\begin{flushright}
Ajia Erin\\
April 14, 2021
\end{flushright}

\tableofcontents
%\listofchanges

\mainmatter
\chapter{前言}
这一章将简单介绍本文档的使用方法、下载地址、更新计划及致谢,想高效率找到自己的遇到的问题,强烈建议浏览目录、点击跳转即可。\\
\begin{itemize}
  \item 第一章~前言:介绍了关于本文档的使用方法等,可略过!
  \item 第二章~初次使用校园网:建议新生参阅,老生可略过!
  \item 第三章~故障处理:本文档的核心内容,可解决你遇到的99.9\%的问题!!!
  \item 第四章~故障报修:介绍了怎样的问题才是清晰的,帮助你将问题提得更好!
  \item 第五章~进阶使用:校园网的高级玩法,满足更多需求,需要一定的基础!
\end{itemize}

\chapter{校园网的初次使用}

\begin{remark}
此处内容适合新生参阅,故障处理请移步第三章  常见故障及处理方法
\end{remark}

\section{手机使用}

\begin{introduction}
  \item 下载手机App
  \item 注册账户/登录
  \item 连接e-LyLg认证上网
\end{introduction}

\section{电脑使用}

\section{pad使用}

\section{其他设备使用}

\chapter{常见故障及处理方法}

\chapter{故障报修流程}

\chapter{进阶使用教程}


\end{document}
